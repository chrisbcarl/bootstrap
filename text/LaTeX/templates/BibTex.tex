% http//www.overleaf.com/latex/templates/ieee-conference-template/grfzhhncsfqn
% pdflatex text\LaTeX\ieee-conference-template.tex
% TODO: toc + hyperlinks - https://www.overleaf.com/learn/latex/Hyperlinks

\documentclass[conference]{IEEEtran}
% \documentclass[a4paper,12pt]{article}
% \usepackage[utf8]{inputenc}
\IEEEoverridecommandlockouts
% The preceding line is only needed to identify funding in the first footnote. If that is unneeded, please comment it out.
\usepackage{cite}
\usepackage{amsmath,amssymb,amsfonts}
\usepackage{algorithmic}
\usepackage{graphicx}
\usepackage{textcomp}
\usepackage{xcolor}
\usepackage{url}
\usepackage{pythonhighlight}  % external, console install if you must, should auto-download

% https://stackoverflow.com/a/3175141
\usepackage{listings}  % https://www.ctan.org/tex-archive/macros/latex/contrib/listings/
% https://www.overleaf.com/learn/latex/Code_Highlighting_with_minted may be an option
% \usepackage{minted}
\definecolor{darkgreen}{rgb}{0,0.6,0}
\definecolor{gray}{rgb}{0.5,0.5,0.5}
\definecolor{mauve}{rgb}{0.58,0,0.82}
\lstset{
    frame=tb,
    language=Java,
    aboveskip=3mm,
    belowskip=3mm,
    showstringspaces=false,
    columns=flexible,
    basicstyle={\small\ttfamily},
    numbers=none,
    numberstyle=\tiny\color{gray},
    keywordstyle=\color{blue},
    commentstyle=\color{darkgreen},
    stringstyle=\color{mauve},
    breaklines=true,
    breakatwhitespace=true,
    tabsize=3
}
% C++ already included, yipee
% https://tex.stackexchange.com/a/83883 - python extra custy


\def\BibTeX{
    {\rm B\kern-.05em{\sc i\kern-.025em b}\kern-.08em T\kern-.1667em\lower.7ex\hbox{E}\kern-.125emX}
}
\begin{document}


\title{
    Conference Paper Title*\\
    {\footnotesize \textsuperscript{*}Note: Sub-titles are not captured in Xplore and should not be used}
    \thanks{Identify applicable funding agency here. If none, delete this.}
}

\author{
    \IEEEauthorblockN{1\textsuperscript{st} Given Name Surname}
    \IEEEauthorblockA{
        \textit{dept. name of organization (of Aff.)} \\
        \textit{name of organization (of Aff.)}\\
        City, Country \\
        email address or ORCID
    }

    \and

    \IEEEauthorblockN{2\textsuperscript{nd} Given Name Surname}
    \IEEEauthorblockA{\textit{dept. name of organization (of Aff.)} \\
    \textit{
        name of organization (of Aff.)}\\
        City, Country \\
        email address or ORCID
    }

}

\maketitle


%%%%%%%%%%%%%%%%%%%%%%%%%%%%%%%%%%%%%%%%%%%%%%%%%%%%%%%%%%%%%%%%%%%%%%%%%%%%%%%%
\begin{abstract}

    This document is a model and instructions for \LaTeX.
    Just type normal paragraphs and you'll be ok.

\end{abstract}
%%%%%%%%%%%%%%%%%%%%%%%%%%%%%%%%%%%%%%%%%%%%%%%%%%%%%%%%%%%%%%%%%%%%%%%%%%%%%%%%


%%%%%%%%%%%%%%%%%%%%%%%%%%%%%%%%%%%%%%%%%%%%%%%%%%%%%%%%%%%%%%%%%%%%%%%%%%%%%%%%
\begin{IEEEkeywords}
component, formatting, style, styling, insert
\end{IEEEkeywords}
%%%%%%%%%%%%%%%%%%%%%%%%%%%%%%%%%%%%%%%%%%%%%%%%%%%%%%%%%%%%%%%%%%%%%%%%%%%%%%%%


%%%%%%%%%%%%%%%%%%%%%%%%%%%%%%%%%%%%%%%%%%%%%%%%%%%%%%%%%%%%%%%%%%%%%%%%%%%%%%%%
\section{Introduction}
\label{sec:intro}

runon paragraph or separate line per sentence or whatever spacing you like is fine
, they will just be `'-concatenated
        Tabs and spaces have no effect on typesetting.
or Math in
    Paper Title
or Abstract.

multiline

does have effect

Backtick ` is the open first quote, ' is the closing. for `quotation by hitting the tilde key', 'single quotes' and "double quotes" dont look great.

Got some tricks here (vspace will give you a br not a bunch of newlines)



\vspace{12pt}

\color{red}Inline red.
\color{black}The color is permanent til you reset it
%%%%%%%%%%%%%%%%%%%%%%%%%%%%%%%%%%%%%%%%%%%%%%%%%%%%%%%%%%%%%%%%%%%%%%%%%%%%%%%%


%%%%%%%%%%%%%%%%%%%%%%%%%%%%%%%%%%%%%%%%%%%%%%%%%%%%%%%%%%%%%%%%%%%%%%%%%%%%%%%%
\section{Section w/ Nested Subsections}
\label{sec:section}

Section w/ Nested Subsections
\subsection{Section w/ Subsections}
\label{subsec:subsec}

% \part{Part - Level 0}
% \label{part:level-0}
% Part - Level 0

\section{Section - Level 1}
\label{sec:level-1}
Section - Level 1

\subsection{Subsection - Level 2}
\label{subsec:level-2}
Subsection - Level 2

\subsubsection{Subsubsection - Level 3}
\label{subsubsec:level-3}
Subsubsection - Level 3

\paragraph{Paragraph - Level 4}
\label{para:level-4}
Paragraph - Level 4

% \subparagraph{Subparagraph - Level 5}
% \label{subpara:level-5}
% Subparagraph - Level 5

%%%%%%%%%%%%%%%%%%%%%%%%%%%%%%%%%%%%%%%%%%%%%%%%%%%%%%%%%%%%%%%%%%%%%%%%%%%%%%%%


%%%%%%%%%%%%%%%%%%%%%%%%%%%%%%%%%%%%%%%%%%%%%%%%%%%%%%%%%%%%%%%%%%%%%%%%%%%%%%%%
\subsection{Code}
\label{subsec:code}

\begin{verbatim}
#include <stdio.h>

int main(int argc, char** argv) {
    printf("Hello, World!\m");
    return 0;
}
\end{verbatim}


\lstset{language=C++}
\begin{lstlisting}
#include <iostream>

int main(int argc, char** argv) {
    std::cout << "Hello, World!" << std::endl;
    return 0;
}
\end{lstlisting}

\lstset{language=Java}
\begin{lstlisting}
// Hello.java
import javax.swing.JApplet;
import java.awt.Graphics;

public class Hello extends JApplet {
    public void paintComponent(Graphics g) {
        g.drawString("Hello, world!", 65, 95);
    }
}
\end{lstlisting}

\begin{python}
def f(x):
    return x
\end{python}

Python inline \pyth|d = {"a": 1, "b": 2}|.

Externally load:

\inputpythonfile{./demo.py}[0][50]


%%%%%%%%%%%%%%%%%%%%%%%%%%%%%%%%%%%%%%%%%%%%%%%%%%%%%%%%%%%%%%%%%%%%%%%%%%%%%%%%
\section*{Section Without Roman Numerals}
\label{sec:noroman}


\subsection{unordered list}
\label{sec:ul}
Symbol
\begin{itemize}
    \item Special characters: \&, \%, \$, \#, \_, \{, \}, \textasciitilde, \textasciicircum, \textbackslash
    \item \LaTeX \BibTeX
\end{itemize}

\subsection{ordered list}
\label{sec:ol}

\begin{enumerate}
    \item \url{http//google.com} - useful for not hyperlinking in bibitem
\end{enumerate}


%%%%%%%%%%%%%%%%%%%%%%%%%%%%%%%%%%%%%%%%%%%%%%%%%%%%%%%%%%%%%%%%%%%%%%%%%%%%%%%%
\section{Hyperlinks, References, Equations, Figures, Tables}
\label{sec:refeq}

\url{https://en.wikibooks.org/wiki/LaTeX/Labels_and_Cross-referencing}

% Internal hyperlink \hyperref[sec:intro]{Introduction}.

% Hyperlink with \href{https://en.wikibooks.org/wiki/LaTeX/Labels_and_Cross-referencing}{alt text}.

Non-hyper with \url{https://en.wikibooks.org/wiki/LaTeX/Labels_and_Cross-referencing}.
Inter-section reference with ~\ref{sec:refeq} ~\ref{sec:noroman}.
Equations reference with ~\eqref{eq:whatever}.
Table reference them Table ~\ref{tab:whatever}.
Figure reference them Fig.~\ref{fig:whatever}.
Citations look like this ~\cite{Citekey-Inproceedings}.

% ch:     chapter
% sec:    section
% subsec: subsection
% fig:    figure
% tab:    table
% eq:     equation
% lst:    code listing
% itm:    enumerated list item
% alg:    algorithm
% app:    appendix subsection

\begin{equation}
a+b=\gamma\label{eq:whatever}
\end{equation}

\begin{table}[htbp]
\caption{Table Type Styles}
\begin{center}
\begin{tabular}{|c|c|c|c|}
\hline
\textbf{Table}&\multicolumn{3}{|c|}{\textbf{Table Column Head}} \\
\cline{2-4}
\textbf{Head} & \textbf{\textit{Table column subhead}}& \textbf{\textit{Subhead}}& \textbf{\textit{Subhead}} \\
\hline
copy& More table copy$^{\mathrm{a}}$& &  \\
\hline
\multicolumn{4}{l}{$^{\mathrm{a}}$Sample of a Table footnote.}
\end{tabular}
\label{tab:whatever}
\end{center}
\end{table}

\begin{figure}[htbp]
    \centerline{\includegraphics[width=\linewidth]{../Layout-dimensions.png}}
    \caption{Example of a figure caption.}
    \label{fig:whatever}
\end{figure}


%%%%%%%%%%%%%%%%%%%%%%%%%%%%%%%%%%%%%%%%%%%%%%%%%%%%%%%%%%%%%%%%%%%%%%%%%%%%%%%%




%%%%%%%%%%%%%%%%%%%%%%%%%%%%%%%%%%%%%%%%%%%%%%%%%%%%%%%%%%%%%%%%%%%%%%%%%%%%%%%%
\subsection{Extra text to pad out}
\paragraph{Positioning Figures and Tables} Place figures and tables at the top and
bottom of columns. Avoid placing them in the middle of columns. Large
figures and tables may span across both columns. Figure captions should be
below the figures; table heads should appear above the tables. Insert
figures and tables after they are cited in the text. Use the abbreviation
Fig.~\ref{fig:whatever}, even at the beginning of a sentence.



%%%%%%%%%%%%%%%%%%%%%%%%%%%%%%%%%%%%%%%%%%%%%%%%%%%%%%%%%%%%%%%%%%%%%%%%%%%%%%%%



\newpage


%%%%%%%%%%%%%%%%%%%%%%%%%%%%%%%%%%%%%%%%%%%%%%%%%%%%%%%%%%%%%%%%%%%%%%%%%%%%%%%%

%MANUAL BIBLIOGRAPHY%

% \begin{thebibliography}{00}
% \bibitem{b1} G. Eason, B. Noble, and I. N. Sneddon, ``On certain integrals of Lipschitz-Hankel type involving products of Bessel functions,'' Phil. Trans. Roy. Soc. London, vol. A247, pp. 529--551, April 1955.
% \bibitem{b2} J. Clerk Maxwell, A Treatise on Electricity and Magnetism, 3rd ed., vol. 2. Oxford: Clarendon, 1892, pp.68--73.
% \bibitem{b3} I. S. Jacobs and C. P. Bean, ``Fine particles, thin films and exchange anisotropy,'' in Magnetism, vol. III, G. T. Rado and H. Suhl, Eds. New York: Academic, 1963, pp. 271--350.
% \bibitem{b4} K. Elissa, ``Title of paper if known,'' unpublished.
% \bibitem{b5} R. Nicole, ``Title of paper with only first word capitalized,'' J. Name Stand. Abbrev., in press.
% \bibitem{b6} Y. Yorozu, M. Hirano, K. Oka, and Y. Tagawa, ``Electron spectroscopy studies on magneto-optical media and plastic substrate interface,'' IEEE Transl. J. Magn. Japan, vol. 2, pp. 740--741, August 1987 [Digests 9th Annual Conf. Magnetics Japan, p. 301, 1982].
% \bibitem{b7} M. Young, The Technical Writer's Handbook. Mill Valley, CA: University Science, 1989.
% \end{thebibliography}


%with LaTeX.bib%
\bibliography{BibTex}        % Entries are in the LaTeX.bib file
\bibliographystyle{ieeetr}
%%%%%%%%%%%%%%%%%%%%%%%%%%%%%%%%%%%%%%%%%%%%%%%%%%%%%%%%%%%%%%%%%%%%%%%%%%%%%%%%

\end{document}



